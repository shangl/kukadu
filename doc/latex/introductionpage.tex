kukadu is a software framework for robotics applications. It was used exensively in the lab of the \href{http://iis.uibk.ac.at}{\tt Intelligent and Interactive} systems group at the University of Innsbruck. It supports several different modules that are essential for robotics applications such as
\begin{DoxyItemize}
\item Robot control (\hyperlink{robotpage}{The robot module})
\item Kinematics and path planning (\hyperlink{kinematicspage}{The kinematics module})
\item Robust control policies (\hyperlink{controlpage}{The control module})
\item Machine learning (\hyperlink{mlpage}{The machine learning module})
\begin{DoxyItemize}
\item Regression
\item Classification
\item Kernel Methods
\item Reinforcement Learning
\end{DoxyItemize}
\end{DoxyItemize}

kukadu is written in C++ and supports usage in combination with \href{http://www.ros.org/}{\tt R\-O\-S}. It provides several general interfaces and implements it with state of the art methods from control, machine learning and robotic manipulation.

This document provides a setup guide (\hyperlink{installationpage}{Installation}) including a recommendation on how to develop kukadu software using \href{https://www.qt.io/ide/}{\tt Qt\-Creator}. Further, several tutorials and an A\-P\-I documentation is given.

Next (\hyperlink{installationpage}{Installation}) 